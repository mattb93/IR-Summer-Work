\chapter{Literature Review}\label{ch:literatureReview}

\section{Textbook}
The textbook \cite{manning2008introduction} introduces the classification problem. That is, given a set of classes, the goal is to determine what subset of those classes a document may belong to. To that end, the textbook describes a number of methodologies for selecting features. These features are then used by one of the classification methods discussed. The primary methods discussed are Na\"{i}ve Bayes, Vector Space Classification, and the Support Vector Machine.

\section{Papers}
The primary paper that has been used as a reference is the final report of the Classification team from last year \cite{cui2015classification}. We have read through this report to understand the progress that the Classification team made last year as well as using the paper to gain an understanding of the task and interactions of the systems that we are using to perform our work. At a high level, the paper described a methodology employed by the team in which they were able to apply the Na{\"i}ve Bayes method to classify sets of data. The team used Apache Mahout machine learning technology to generate Na{\"i}ve Bayes classifiers to make predictions for new data. The biggest difference from last year's work to this year's work is that we will be primarily using Apache Spark, and so while we will be able to reference their work, we are using entirely different technologies and will need to modify our approach accordingly.

Initially, we attempted an approach where we would issue queries to Solr to build a set of training data for a classifier. However, we were informed by the GRAs that this approach would not work because the Solr API was not exposed on the cluster, and so we would not be able to access it. When we started looking into new approaches, we were advised to look at Frequent Pattern Mining (FPM). We have looked at a paper by Han et al.\ \cite{han2000mining}, which presents a novel frequent pattern tree (FP-tree) structure and FP-tree based mining method called FP-growth. This method allows for mining the complete set of frequent patterns by pattern fragment growth. In this paper they also demonstrate that their new method is an order of magnitude faster than the typical apriori algorithm. The paper goes on to compare the performance against other popular data mining algorithms and discusses the use of the algorithm on some large industrial databases. We are currently making use of the FP-growth algorithm and data structure in our working classification prototype.