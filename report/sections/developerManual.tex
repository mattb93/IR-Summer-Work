\chapter{Developer Manual}\label{ch:developerManual}
%\studentComment{more concrete instructions such as on how the code is being developed, module overview, and how the data was processed, etc.}
%\studentComment{collaborate more on the design and implementation steps}

The developer manual will grow and evolve with the project. It documents the tools we are currently using, not tools we are considering for future steps.

\section{Algorithms}

\subsection{Frequent Pattern Mining}
Our methodology for constructing training data is based on a frequent pattern mining algorithm \cite{han2000mining}. \href{https://spark.apache.org/docs/1.5.0/mllib-frequent-pattern-mining.html}{Spark.mllib} version 1.5 provides a parallel implementation of FP-growth, a popular algorithm for mining frequent itemsets. 
\subsection{Logistic Regression}
Logistic regression is a popular method to predict a binary response. It is a special case of \href{https://en.wikipedia.org/wiki/Generalized_linear_model}{Generalized Linear models} that predicts the probability of the outcome. For more background and more details about the implementation, refer to the documentation of the \href{https://spark.apache.org/docs/1.5.0/mllib-linear-methods.html#logistic-regression}{logistic regression in spark.mllib}.

\foxComment{What might a developer have to change in the future?

Might there be scaling concerns?

Might some methods need to be changed for different collections?

Are there different concerns for webpages than tweets?}
\section{Environment Setup}

\foxComment{Should this section be in the user manual?}

\subsection{Dependencies}

\subsubsection{Java}
You will want the latest version of Java, which at the time of this writing is Java Version 8 Update 73. Download and installation instructions for your environment can be found at:

\href{https://java.com/en/download/}{https://java.com/en/download/}

The Hadoop cluster we are working with is currently running Java Version 7 Update 99. This version of Java is capable of performing everything that we require for this work. However, due to security concerns among other things we recommend using the latest stable version of Java if setting up a cluster from scratch.

\subsubsection{Python}
You will need Python 2.6.6 (our work should be compatible with newer versions of the 2.x.x line, but we have not tested it). Download and installation instructions for your environment can be found at:

\href{https://www.python.org/downloads/}{https://www.python.org/downloads/}

While the cluster that we have implemented our solution on is using Python 2.6.6, we recommend using the latest stable version of the Python 2.x.x line if setting up a cluster from scratch.

\subsection{Apache Spark}
This system is built on Apache Spark 1.5.0, which can be downloaded at:

\href{http://spark.apache.org/downloads.html}{http://spark.apache.org/downloads.html}

Any newer versions of Spark should also be compatible for the foreseeable future. So if setting up a cluster from scratch, we recommend installing the latest stable version of Spark.

% Here is the introduction. The next chapter is chapter~\ref{ch:ch2label}.


% a new paragraph


% \section{Examples}
% You can also have examples in your document such as in example~\ref{ex:simple_example}.
% \begin{example}{An Example of an Example}
%   \label{ex:simple_example}
%   Here is an example with some math
%   \begin{equation}
%     0 = \exp(i\pi)+1\ .
%   \end{equation}
%   You can adjust the colour and the line width in the {\tt macros.tex} file.
% \end{example}

% \section{How Does Sections, Subsections, and Subsections Look?}
% Well, like this
% \subsection{This is a Subsection}
% and this
% \subsubsection{This is a Subsubsection}
% and this.

% \paragraph{A Paragraph}
% You can also use paragraph titles which look like this.

% \subparagraph{A Subparagraph} Moreover, you can also use subparagraph titles which look like this\todo{Is it possible to add a subsubparagraph?}. They have a small indentation as opposed to the paragraph titles.

% \todo[inline,color=green]{I think that a summary of this exciting chapter should be added.}